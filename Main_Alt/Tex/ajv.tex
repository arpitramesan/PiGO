\documentclass[svgnames,9pt]{beamer}
%\usepackage[latin1]{inputenc}
\usetheme{Warsaw}
\setbeamertemplate{headline}{}
\setbeamerfont{footline}{size=\fontsize{6}{8}\selectfont}
\makeatletter
\expandafter\def\expandafter\insertshorttitle\expandafter{%
        \insertshorttitle\hfill%
        \insertframenumber\,/\,\inserttotalframenumber}
\setbeamertemplate{frametitle}
{%

        \nointerlineskip%
        \vskip-2pt%
        \hbox{\leavevmode
                \advance\beamer@leftmargin by -12bp%
                \advance\beamer@rightmargin by -12bp%
                \beamer@tempdim=\textwidth%
                \advance\beamer@tempdim by \beamer@leftmargin%
                \advance\beamer@tempdim by \beamer@rightmargin%
                \hskip-\Gm@lmargin\hbox{%
                        \setbox\beamer@tempbox=\hbox{\begin{minipage}[b]{\paperwidth}%
                                        \vbox{}\vskip.50ex% NEW: original \vskip-.75ex
                                        \leftskip0.3cm%
                                        \rightskip0.3cm plus1fil\leavevmode
                                        \insertframetitle%
                                        \ifx\insertframesubtitle\@empty%
                                        \strut\par%
                                        \else
                                        \par{\usebeamerfont*{framesubtitle}{\usebeamercolor[fg]{framesubtitle}\insertframesubtitle}\strut\par\vskip2ex}% NEW: added \vskip2ex
                                        \fi%
                                        \nointerlineskip
                                        \vbox{}%
                        \end{minipage}}%
                        \beamer@tempdim=\ht\beamer@tempbox%
                        \advance\beamer@tempdim by 2pt%
                        \begin{pgfpicture}{0pt}{0pt}{\paperwidth}{\beamer@tempdim}
                                \usebeamercolor{frametitle right}
                                \pgfpathrectangle{\pgfpointorigin}{\pgfpoint{\paperwidth}{\beamer@tempdim}}
                                \pgfusepath{clip}
                                \pgftext[left,base]{\pgfuseshading{beamer@frametitleshade}}
                        \end{pgfpicture}
                        \hskip-\paperwidth%
                        \box\beamer@tempbox%
                }%
                \hskip-\Gm@rmargin%
        }%
        \nointerlineskip
        \vskip-0.2pt
        \hbox to\textwidth{\hskip-\Gm@lmargin\pgfuseshading{beamer@topshade}\hskip-\Gm@rmargin}
        \vskip-2pt
}
\makeatother
\addtobeamertemplate{frametitle}{\vspace*{0.2cm}}{\vspace*{0.3cm}}

\title[Short title]{\huge \textbf{Python coding for geospatial processing in web-based mapping applications}  } %Python coding for geospatial processing in web-based mapping applications
\author[Group 14]{}

\date{}

\begin{document}
        \begin{frame}

                \begin{figure}[h!]
                        \includegraphics[scale=0.5]{stthomaskannur.png}
                \end{figure}

                \begin{center}     
                        \begin{minipage}[b]{1.0\textwidth}
                                \centering
                                Department of Computer Science and Engineering\\
                                St. Thomas College of Engineering and Technology\\Mattannur
                        \end{minipage}%
                \end{center}

                \titlepage

                \begin{minipage}[t]{0.5\textwidth}
                        \vspace{-3cm}
                        \begin{flushleft}
                                { \textit{Author}:\vspace*{0.2cm}\\STUDENT 1(STM19CS017)-ARPIT RAMESAN\\STUDENT 2(STM19CS024)-JAREESH ISMAIL CK\\STUDENT 3(STM19CS057)-VAISHNAVI RAJEEVAN }
                                
                                 %students name
                        \end{flushleft}
                \end{minipage}%
                %
                \begin{minipage}[t]{0.5\textwidth}
                        \vspace{-2cm}
                        \begin{flushright}
                                {\textit{Supervisor}:\vspace*{0.1cm} \\ Mr.JITHIN S} %supervisor name
                        \end{flushright}   
                \end{minipage}%

                \vspace{-0.5cm}
                \begin{center}     
                        \begin{minipage}[b]{0.5\textwidth}
                                \centering 
                                \small Academic Year \\ 2021-22\\ \today
                        \end{minipage}%
                \end{center}   
        \end{frame}
\section{OUTLINE}
\begin{frame}{OUTLINE}
        \tableofcontents
\end{frame}
\section{INTRODUCTION}
\begin{frame}{INTRODUCTION TO DOMAIN}
Web-based mapping applications are effective in providing
simple and accessible interfaces for geospatial information,
and often include large spatial databases and advanced an-
alytical capabilities. Perhaps the most familiar is Google
Maps [Ggl] which provides access to terabytes of maps,
aerial imagery, street address data, and point-to-point routing
capabilities. Descriptions are included herein of several Web-
based applications that focus on energy and environmental data
and how their back-end geoprocessing services were built with
Python.
\end{frame}

\section{PROBLEM IDENTIFICATION}
\begin{frame}
{PROBLEM IDENTIFICATION}
\end{frame}
\section{LITERATURE SURVEY 1}
\begin{frame}{LITERATURE SURVEY 1}

\end{frame}
\section{LITERATURE SURVEY 2}
\begin{frame}{LITERATURE SURVEY 2}


\end{frame}
\section{EXISTING SYSTEM}
\begin{frame}{EXISTING SYSTEM}

\end{frame}
\section{PROPOSED SYSTEM} 
\begin{frame}{PROPOSED SYSTEM}

\end{frame}
\section{CONCLUSION}
\begin{frame}{ CONCLUSION }     
\begin{itemize}
Python is the de-facto standard scripting language in both
the open source and proprietary GIS world. Most, if not all,
of the major GIS software systems provide Python libraries
for system integration, analysis, and automation, including
ArcGIS, GeoPandas [GeoP], geoDjango [geoD], GeoServer,
GRASS, PostGIS, pySAL [pySAL], and Shapely [Shp]. Some
of these systems, such as ArcGIS and geoDJango, provide
frameworks for web-based mapping applications different
from the approach we described in the SOFTWARE ENVI-
RONMENT section. While it is outside the scope of this paper
to discuss the merits of these other approaches, we recommend
considering them as alternatives when planning projects.
The examples in this paper include vector and raster data,
as well as code for converting projections, creating buffers,
retrieving features within a specified area, computing areas
and lengths, computing a raster-based model, and exporting
raster results in GeoTIFF format. All examples are written in
Python and run within the OGC-compliant WPS framework
provided by PyWPS.
\item 
\item 
\end{itemize}
\end{frame}
\section{REFERENCES}
\begin{frame}{REFERENCES}

\begin{thebibliography}{99}
            [Arg13] Argonne National Laboratory, Energy Zones Study: A Comprehen-
sive Web-Based Mapping Tool to Identify and Analyze Clean Energy
Zones in the Eastern Interconnection, ANL/DIS-13/09, September
2013. Available at https://eispctools.anl.gov/document/21/file
[Btsrp] http://getbootstrap.com
[DOI12] U.S. Department of the Interior, Bureau of Land Management,
and U.S. Department of Energy, Final Programmatic Environmental
Impact Statement for Solar Energy Development in Six Southwestern
States, FES 12-24, DOE/EIS-0403, July 2012. Available at http:
//solareis.anl.gov/documents/fpeis
[Erc] http://bogi.evs.anl.gov/erc/portal
[Ezmt] http://eispctools.anl.gov
[GDAL] http://www.gdal.org
[geoD] http://geodjango.org
[GeoP] http://geopandas.org
[Ggl] http://maps.google.com
[GRASS] http://grass.osgeo.org
[Gsrvr] http://geoserver.org
[NGA] http://earth-info.nga.mil/GandG/wgs84/web_mercator/index.html
[OGP] http://www.epsg.org
[OpLyr] http://openlayers.org
[PGIS] http://postgis.net/docs/manual-2.0/reference.html
[pySAL] http://pysal.readthedocs.org/en/v1.7
[PyWPS] http://pywps.wald.intevation.org
[RoR] http://rubyonrails.org
[Sen] http://www.sencha.com/products/extjs
[Shp] http://pypi.python.org/pypi/Shapely
[Sol] http://solarmapper.anl.gov
[Sol13] Kuiper, J., Ames, D., Koehler, D., Lee, R., and Quinby, T., "Web-
Based Mapping Applications for Solar Energy Project Planning," in
Proceedings of the American Solar Energy Society, Solar 2013 Con-
ference. Available at http://proceedings.ases.org/wp-content/uploads/
2014/02/SOLAR2013_0035_final-paper.pdf.
[Ub] http://www.ubuntu.com
[VM] http://www.vmware.com
\end{thebibliography}
\end{frame}
\end{document}
